\documentclass[11pt]{article}
\usepackage{graphicx}
\newcommand{\mylabel}[1]{\textnormal{\textsf{[#1]}}
                                                                                
\label{#1}}
\renewcommand{\mylabel}[1]{\label{#1}}

\setlength{\oddsidemargin}{.25in}
\setlength{\evensidemargin}{.25in}
\setlength{\textwidth}{6in}
\setlength{\topmargin}{-0.4in}
\setlength{\textheight}{8.5in}

%
% mathify-- ensure argument is in math mode
%

\newcommand{\mathify}[1]{\ifmmode{#1}\else\mbox{$#1$}\fi}

\newcommand{\brk}[1]{\langle #1 \rangle}
%
% Impressive box to go at the top of the first page
%

\def\subjnum{601.745}
\def\subjname{Advanced Topics in Applied Cryptography}


\newcommand{\handout}[5]{
  \renewcommand{\thepage}{#1-\arabic{page}}
  \noindent
  \begin{center}
    \framebox{
      \vbox{
        \hbox to 5.78in { {\bf \subjnum: \subjname} \hfill #2 }
        \vspace{4mm}
        \hbox to 5.78in { {\Large \hfill #5  \hfill} }
        \vspace{2mm}
        \hbox to 5.78in { {\it #3 \hfill #4} }
        }
      }
  \end{center}
  \vspace*{4mm}
  }

\newcommand{\lecture}[4]{\handout{#1}{#2}{Lecturer:
    #3}{Scribe: #4}{Lecture #1}}

\newcommand{\homework}[4]{\handout{#1}{#2}{#3}{#4}{Homework #1}}
\newcommand{\abs}[1]{\mathify{\left| #1 \right|}}
\renewcommand{\Pr}[1]{\mathify{\mbox{Pr}\left[#1\right]}}
\newcommand{\Exp}[1]{\mathify{\mbox{Exp}\left[#1\right]}}
\newcommand{\set}[1]{\mathify{\left\{ #1 \right\}}}
\newcommand{\cset}[2]{\set{#1\ :\ #2}}  % a conditional notation to define sets
\newcommand{\lset}[2]{\set{#1,\ldots,#2}} % set {from,...,to}
\newcommand{\suchthat}{\vert}
\newcommand{\st}{\suchthat}
\newcommand{\length}[1]{\left\vert #1 \right\vert}
\newtheorem{theorem}{Theorem}
\newtheorem{proofsketch}{Proof sketch}
\newtheorem{corollary}[theorem]{Corollary}
\newtheorem{lemma}[theorem]{Lemma}
\newtheorem{observation}[theorem]{Observation}
\newtheorem{proposition}[theorem]{Proposition}
\newtheorem{definition}[theorem]{Definition}
\newtheorem{claim}[theorem]{Claim}
\newtheorem{fact}[theorem]{Fact}
\newtheorem{assumption}[theorem]{Assumption}
\newcounter{problem}
\newcounter{subproblem}
\renewcommand{\theproblem}{\arabic{problem}}
\renewcommand{\thesubproblem}{\arabic{subproblem}}
\newenvironment{problem} {\stepcounter{problem} \textbf{Problem
    \theproblem}:} {\vspace{.1in}}
\newenvironment{subproblem} {\stepcounter{subproblem} 
    \thesubproblem :} {}
\newenvironment{solution} {\textbf{Solution
    \theproblem}:} {\vspace{.1in}}
\newcommand{\rep}[1]{\left\langle #1 \right\rangle}
\newcommand{\Gen}{\mathsf{Gen}}
\newcommand{\Tag}{\mathsf{Tag}}
\newcommand{\Vfy}{\mathsf{Vrfy}}
\newcommand{\Enc}{\mathsf{Enc}}
\newcommand{\Dec}{\mathsf{Dec}}
\newcommand{\Group}{\mathbb{G}}
\newcommand{\Zq}{\mathbb{Z}_q}
\newcommand{\pk}{\mathsf{pk}}
\newcommand{\sk}{\mathsf{sk}}
\newcommand{\Z}{\mathbb Z}
\newcommand{\N}{\mathbb N}
\newcommand{\bitset}{\{0,1\}^*}
\newcommand{\ma}{\mathcal{A}}
\newcommand{\mb}{\mathcal{B}}
\usepackage{url,subfigure,pstricks,ifthen,calc,xspace,graphicx,wrapfig,mdwlist}
\usepackage{amsmath,amsfonts}
\psset{unit=0.1in}
\newcommand{\str}{\mbox{\LARGE\strut}}

\begin{document}
\handout{Assignment 1}{January 31, 2018}{Instructor: Matthew Green} {Due: 11:59pm, February 12} {Assignment 1}

\bigskip \bigskip \noindent
{\Large Name:} \underline{~~~~~~~~~~~~~~~~~~~~~~~~~~~~~~~~~~~~~~~~~~~~~~~~~~~~~~~~~~~~~~~~~~~~~~~~~~~~~~~~~~~~~~~~~~~~~~~~~~~~~}

\bigskip
\noindent
{\bf The assignment should be completed individually. You {are} permitted to use the Internet and any printed references.  

\medskip \noindent
%Please note: this is only one part of a two-part assignment.  Instructions on obtaining Part 2 are described below.  
Please submit the completed assignment via Blackboard.}

\newcommand{\adversary}{{\cal A}}
\newcommand{\newadversary}{{\cal B}}

\bigskip \noindent 
\begin{problem} {\bf PRFs as MACs.}
%Sketch a proof of security (or break) the following authenticated encryption constructions. Use the definition of secure authenticated encryption we discussed in class\footnote{See \url{https://cseweb.ucsd.edu/~mihir/papers/oem.html}.} Your proofs do not need to be concrete or formal, but should spell out a full reduction to the hardness of an underlying {\sf SUF-CMA} MAC scheme and {\sf IND-CPA} encryption scheme.

\medskip \noindent
Let $f: \{0,1\}^{\kappa} \times \{0,1\}^* \rightarrow \{0,1\}^{\ell}$ be a pseudorandom function (PRF) family, such that $f_k(m)$ represents the evaluation with key $k$ at point $m$, and the result is an $\ell$ bit string.\footnote{For a formal definition, see {\em e.g.,} \url{https://cseweb.ucsd.edu/~mihir/cse207/w-prf.pdf} and specifically the security game in Definition 3.4.1.}
%
Sketch an informal proof that if $f$ is pseudorandom (and $\ell$ is long enough, say 128 bits), then $f_k(m)$ is a secure MAC on key $k$ and message $m$ in the {\sf SUF-CMA} definition.\footnote{See {\em e.g.,} \url{https://www.cs.jhu.edu/~astubble/dss/ae.pdf}.}

\medskip \noindent
{\em To help you with this, we will sketch out parts of the theorem and proof below.}

\begin{theorem}
{\em
Let $f: \{0,1\}^{\kappa} \times \{0,1\}^* \rightarrow \{0,1\}^{\ell}$ be a PRF family. Then the construction $f_k(m)$ is an {\sf SUF-CMA} MAC.}

\end{theorem}

\begin{proofsketch}
{\em
Our proof proceeds as follows. Let us assume by contradiction that $f_k(m)$ is not a secure MAC scheme, {\em i.e.,} that there exists some p.p.t. adversary $\adversary$ that wins the {\sf SUF-CMA} MAC game with non-negligible advantage.\footnote{As a reminder, in the {\sf SUF-CMA} game the adversary is allowed to query an oracle on any number of messages $m_i$, and receives MACs of the form $T = f_k(m_i)$ for each query. At the end of the game it wins if it outputs a pair $(m^*, T^*)$ such that no previous oracle query (resp. response) was $(m^*, T^*)$ and $T^* = f_k(m^*)$. The advantage of $\adversary$ is the probability that it succeeds in this game.} Then we show that there exists a p.p.t. algorithm $\newadversary$ that wins the PRF game, {\em i.e.,} that distinguishes the function $f$ from a random function with non-negligible advantage.

$\newadversary$ operates as follows. It plays the {\sf PRF} game with a challenger. It also runs $\adversary$ internally, and interacts with it as in the {\sf SUF-CMA} game. When $\adversary$ queries the MAC oracle on a message $m_i$, $\newadversary$ answers the query as follows:

\medskip \noindent
{\bf Fill in this part.}

\medskip \noindent
When $\adversary$ outputs the ``forgery'' pair $(m^*, T^*)$ (such that, by the definition of $\adversary$, with non-negligible probability $T^* = f_k(m^*)$), $\newadversary$ does the following, and outputs a bit $b$ as its guess in the PRF game.

\medskip \noindent
{\bf Fill in this part.}

\medskip \noindent
We argue that if the PRF oracle implements a random function, then:

\medskip \noindent
{\bf Here explain why $\newadversary$ is able to distinguish whether the oracle implements a PRF or a random function with non-negligible advantage.}


\medskip \noindent
This completes the proof.}
\end{proofsketch}

\end{problem}

\begin{problem} {\bf Encrypt-and-MAC.} 

\medskip \noindent Let $k_1, k_2$ be two secret keys. Let {\sf Encrypt}$(k_1, M)$ represent encryption of $M$ using a {\sf IND-CPA} encryption scheme under key $k_1$. Let ${\sf MAC}(k_2, M')$ represent the computation of a (deterministic) MAC on message $M'$ using key $k_2$. Let us define the following authenticated encryption scheme:
$$C = \mathsf{Encrypt}(k_1, M) \| {\sf MAC}(k_2, M) $$
In class we discussed how this scheme is not secure, because there is a simple attack that breaks the {\sf IND-CPA} (and hence the {\sf IND-CCA}) security of the scheme. Despite this, I want you to {\em attempt} to sketch the reduction proof showing that the above scheme is {\sf IND-CCA}, similar to the one that we discussed in class. Tell me where in the proof your attempt breaks down. This can be a quick explanation, not a full proof.
\end{problem}


\begin{problem} {\bf Encrypt-and-Counter-MAC.} 

\medskip \noindent
Many versions of the {\tt ssh} protocol use a variant of the following scheme. Let $k_1, k_2$ be two secret keys. Let {\sf Encrypt}$(k_1, M)$ represent encryption of $M$ using a {\sf IND-CPA} encryption scheme under key $k_1$. Let $f_{k_2}(M')$ represent the computation of a pseudorandom function on message $M'$ using key $k_2$ (this acts as a MAC; see Problem 1). 

\medskip \noindent
Finally, let $i$ be a counter value that begins with $i=0$ on the first message encrypted, and increments for every subsequent message  ({\em i.e.,} you can trust that $i$ will never repeat). The overall authenticated encryption algorithm for message $M$ using counter $i$ is:
$$C =  \mathsf{Encrypt}(k_1, M) \| f_{k_2}(M \| i) $$
Is this scheme a secure {\sf IND-CCA} authenticated encryption scheme? If so, sketch a proof that this is true. If not, demonstrate an attack on the scheme that wins the game with non-negligible probability.

\medskip \noindent
{\em To help you with this, we will sketch out parts of the theorem and proof below.}

\begin{theorem}
{\em
Let $f: \{0,1\}^{\kappa} \times \{0,1\}^* \rightarrow \{0,1\}^{\ell}$ be a PRF family, and let $({\sf Encrypt}, {\sf Decrypt})$ be an {\sf IND-CPA}-secure encryption scheme. Then the construction above is an {\sf IND-CCA} encryption scheme.}

\end{theorem}

\begin{proofsketch}
{\em
Our proof proceeds as follows. Let us assume by contradiction that the scheme above is {\em not} {\sf IND-CCA} secure {\em i.e.,} that there exists some p.p.t. adversary $\adversary$ that wins the {\sf IND-CCA} game with non-negligible advantage.
Then we show that there exists a p.p.t. algorithm $\newadversary$ that wins either the {\sf IND-CPA} game against the encryption scheme, or the {\sf PRF} game, {\em i.e.,} that distinguishes the function $f$ from a random function with non-negligible advantage.

$\newadversary$ operates as follows. It plays the {\sf IND-CPA} game with a challenger. It also runs $\adversary$ internally, and interacts with it as in the {\sf IND-CCA} game. First, each time $\adversary$ requests the encryption of a message $m_i$, $\newadversary$ requests the encryption of $m_i$ from the {\sf IND-CPA} challenger, to obtain $c_i$. $\newadversary$ then generates a random string $T_i \leftarrow \{0,1\}^{\ell}$ and records $(m_i, c_i, T_i)$ in a table. It returns $C_i = c_i \| T_i$. When $\adversary$ queries for the decryption of some $C'_i$, $\newadversary$ parses this as $C'_i = c'_i \| T'_i$ and checks its table for an entry of the form $(m'_i, c'_i, T'_i)$. If one is present, it returns $m'_i$. Otherwise it returns $\bot$ to $\adversary$. 

\medskip \noindent
{\bf Explain the rest of the operation of $\newadversary$ here.}


%Each time $\adversary$ requests the decryption of a ciphertext $C'_i$, $\newadversary$ checks whether $C'_i = C_j$ for some $i,j$ and if so returns $m_i$. Otherwise $\newadversary$ returns $\bot$. When $\adversary$ outputs $m_0, m_1$, $\newadverary$ sends these to the {\sf IND-CPA} challenger 
\medskip \noindent
{\bf Now argue that as long as $\adversary$ cannot distinguish the random strings $T_i$ from correctly-generated tags $f_k(m_i)$ except with negligible probability, then $\newadversary$ succeeds against the {\sf IND-CPA} game with non-negligible probability.}

\medskip \noindent
{\bf Finally, present a second proof that $\adversary$ cannot distinguish these strings.}


\end{problem}

\end{document}





